\chapter{Introduction}

Predicting stock prices remains one of the most challenging applications of time series analysis, primarily due to the Efficient Market Hypothesis (EMH) proposed by Eugene Fama. 
According to EMH, asset prices reflect all available information, making it inherently difficult to achieve consistent predictive accuracy (Fama, 1970). 
While there is an abundance of empirical research focused on stock price prediction (Kara et al., 2011), the literature specifically addressing cryptocurrency price predictions 
remains significantly less developed.


Cryptocurrencies, led by Bitcoin, have emerged as a significant force in the financial markets. 
Bitcoin, often referred to as the flagship of the cryptocurrency world, boasts a market capitalization close to one trillion dollars as of July 2024. 
Alongside Bitcoin, other notable cryptocurrencies such as Ethereum and Solana operate on proof-of-stake protocols, with market capitalizations of approximately 
400 billion and 100 billion dollars, respectively. The digital asset market, encompassing both tangible and intangible assets, is projected to reach a staggering
 24 trillion dollars by 2025 (Muddasir et al., 2020).

 
Given the substantial trading volume and increasing market relevance of cryptocurrencies, this study aims to explore whether their prices can be predicted despite the principles of EMH. 
By focusing on Bitcoin, Ethereum, and Solana, this thesis will employ a multifaceted feature dataset to examine the potential for accurate cryptocurrency price predictions. 
This investigation not only contributes to the academic discourse but also holds practical significance in enhancing our understanding of the cryptocurrency sector.
