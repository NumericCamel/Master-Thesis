\chapter*{Literature Review}
\addcontentsline{toc}{chapter}{Literature Review}
\setcounter{chapter}{2}

Predicting cryptocurrency prices using machine learning (ML) and deep learning (DL) methodologies is a relatively underexplored area in empirical research compared to stock market forecasting. To bridge this gap, our literature review will also consider past works and empirical studies on stock price prediction to glean insights applicable to cryptocurrency forecasting.
We will delve into the theoretical foundations of stock price predictability, examining previous research efforts by analyzing their methodologies and the ML and DL techniques they employed. These studies utilized sophisticated feature sets, including various market variables, and we will review the literature to understand which feature sets were used and the rationale behind their selection.
Moreover, this review aims to illuminate the impact of marketing-related variables on the predictive power of both stock and cryptocurrency markets. Specifically, we will explore the influence of social media sentiment, such as Reddit discussions, on cryptocurrency prices. By doing so, this comprehensive review seeks to provide a deeper understanding of the methodologies and variables that enhance the predictability of financial markets.

\section{Efficient market hypothesis}

Eugene Fama's Efficient Market Hypothesis (EMH) asserts that stock prices reflect all available information, implying that they always trade at their fair value. Fama (1970) argues that since all new information is immediately incorporated into stock prices, consistently predicting market movements or outperforming the market through traditional stock-picking is essentially impossible. This underpins the argument for passive index fund investing, which aims to match market returns rather than exceeding them.

Behavioral finance challenges Eugene Fama's Efficient Market Hypothesis by arguing that psychological factors and irrational behavior of investors can lead to market inefficiencies. Scholars like Daniel Kahneman and Amos Tversky, who develop prospect theory, demonstrate that cognitive biases such as overconfidence and loss aversion significantly influence investor decisions, often leading to predictable and systematic errors. Kahneman and Tversky (1979) mention that these biases cause stock prices to deviate from their true values, creating opportunities for superior returns through strategic trading, contrary to EMH’s assertion that such opportunities are fleeting or non-existent. Behavioral finance thus provides a framework to understand why and how markets might not be entirely efficient.

In 2022, Ho-Jun Kang and his colleagues conduct research to investigate the presence of the Efficient Market Hypothesis (EMH) in the cryptocurrency market. Their study involves testing 893 cryptocurrencies, and the results reveal that only a small fraction of these currencies adhere to the EMH. Specifically, Kang et al. (2022) find that only 54 cryptocurrencies (6\%) follow the weak-form EMH, and just 24 (3\%) adhere to the semi-strong-form EMH. These findings suggest that the cryptocurrency market demonstrates limited efficiency in information processing. Moreover, the study concludes that most cryptocurrencies do not incorporate past prices or new information into their market prices.

In 2020, Vu Le Tran authors a paper examining the Efficient Market Hypothesis (EMH) within the cryptocurrency market. Tran (2020) concludes that market efficiency is highly variable over time, particularly noting significant inefficiencies before 2017. Tran observes that, over time, the cryptocurrency market is becoming increasingly efficient. Among the cryptocurrencies tested, Litecoin emerges as the most efficient, while Ripple is identified as the least efficient.

\section{Past attempts at predicting the stock/crypto currency market} 

In the study conducted by Kara, an artificial neural network (ANN) and support vector machine (SVM) are employed to predict stock price movements on the Istanbul Stock Exchange. The independent variable in this research is a binary indicator reflecting whether the stock price will move up or down the following day. Kara (2011) finds that the SVM, particularly with a polynomial activation function, outperforms all other algorithms, including ANN and backpropagation network (BPN), achieving an accuracy of 71.5\%. The feature set for this study comprises various technical analysis (TA) indicators such as the Moving Average Convergence Divergence (MACD), Moving Average (MA), and the stochastic oscillator \%K (K\%).

In another study focusing on trend deterministic data for stock price prediction, a classification model is used to forecast the up or down movement of stock prices. Patel (2015) mentions that this research incorporates a feature set consisting of binary variables indicating whether a technical indicator suggests an upward or downward trend. The highest performing model in this study is the random forest, which achieves an accuracy of 83.5\%. However, a noted limitation of this approach is the binary nature of the technical indicators. The study suggests that incorporating additional levels to represent the degree of movement, such as 'slightly up', 'slightly down', and 'barely down', could enhance the model's accuracy.

In 2020, Chen conducts a study to predict Bitcoin prices using various machine learning methods, including logistic regression and long short-term memory (LSTM) networks. Chen (2020) finds that by utilizing 5-minute interval price data, the model achieves an accuracy of 66\%, outperforming more complex neural network models. The feature set in this study is comprehensive, incorporating not only Bitcoin price data but also external factors such as gold spot prices, property and network data, as well as trading and market information.

In another study, Weng (2018) attempts to predict short-term stock prices using ensemble methods. The feature set in this research is diverse, comprising historical stock prices, well-known technical indicators, sentiment scores derived from published newspaper articles, trends in Google searches, and the number of visits to Wikipedia pages. The study demonstrates impressive results, predicting the next day's stock prices with a mean absolute percentage error (MAPE) of less than 1.5\%. The best-performing algorithms in this research are boosted decision trees, including XGBoost and AdaBoost.

Usami et al. conduct a study to predict the Karachi Stock Exchange (KSE) using various machine learning algorithms. They employ a classification model to forecast whether the market will go up or down. The feature set for this study is extensive, including oil rates, gold and silver rates, interest rates, foreign exchange (FEX) rates, news and social media feeds, simple moving averages (SMA), and autoregressive integrated moving average (ARIMA) data. Usami et al. (Year) find that the best performing model is the multilayer perceptron (MLP), a type of artificial neural network (ANN), alongside support vector regression (SVR).

In 2022, Mailagaha Kumbure et al. conduct a comprehensive literature review on the application of machine learning and data used for stock market forecasting. This review examines a total of 138 articles related to machine learning in stock markets, providing a detailed overview of the models, markets, and feature sets used in these studies. Mailagaha Kumbure et al. (2022) highlight that the most used machine learning methods are neural networks, support vector machines/support vector regression (SVM/SVR), and fuzzy theories. Additionally, they note that most of these papers incorporate technical indicators in their feature sets.

\section{Social Media and Sentiment Analysis in the Role of Predicting Stock/Crypto Prices}

The influence of social media on Bitcoin prices has been a topic of significant interest in recent research. Feng Mai's 2018 study employs textual analysis and vector error corrections to demonstrate a clear link between social media sentiment and Bitcoin price movements. Mai (2018) shows that bullish posts on social media platforms are associated with higher future Bitcoin prices. This suggests that social media sentiment is a valuable predictor of Bitcoin price fluctuations, highlighting the impact of public opinion and social discourse on cryptocurrency markets.

In addition to social media, online search activity also correlates with Bitcoin price movements. Kristoufek (2013) analyzes Google Trends and Wikipedia page visits, finding strong correlations between these data points and Bitcoin prices. This research suggests that increased online searches and Wikipedia activity, reflecting public interest and awareness, can significantly influence Bitcoin market trends. Complementing these findings, Wesley S. Chan's 2003 study on stock market prediction through news sentiment reveals that positive newspaper headlines often lead to overvaluation of stocks, while negative headlines result in undervaluation. Chan (2003) further notes that this sentiment effect is more pronounced in smaller market capitalization stocks and that investors typically react slowly to sentiment changes. Together, these studies underscore the significant role of public sentiment, whether expressed through social media, search activity, or news headlines, in influencing financial markets.

The predictive power of social media sentiment on cryptocurrency prices has been further explored in recent studies. Olivier Kraaijeveld's 2020 research focuses on the influence of Twitter sentiment on the returns of major cryptocurrencies. Kraaijeveld (2020) concludes that Twitter sentiments indeed have predictive power over cryptocurrency prices, utilizing a lexicon-based sentiment analysis. The study highlights that news disseminated through Twitter can rapidly alter investor sentiments, leading to immediate and significant price movements. This finding emphasizes the crucial role of real-time sentiment analysis in anticipating market trends and price fluctuations in the volatile cryptocurrency market.

Similarly, news sentiment shows a notable impact on Bitcoin prices. Lavinia Rognone's 2020 study analyzes the effect of unscheduled news on Bitcoin compared to traditional currencies using intra-day data from January 2012 to November 2018. Rognone (2020) finds that Bitcoin often reacts positively to news, whether positive or negative, indicating a high level of enthusiasm among investors towards Bitcoin, unlike traditional stock markets. However, specific negative news, such as reports of fraud and cyber-attacks, have adverse effects on Bitcoin prices. The study utilizes RavenPack's real-time news data and employs a Vector Auto-Regressive Exogenous (VARX) model for the analysis. In parallel, Wasit Khan's 2020 research combines social media and news sentiment to predict stock market movements, using a dataset from Twitter and Yahoo Finance. Khan (2020) demonstrates that their predictive model achieves an accuracy of 80\% after filtering out spam tweets, underscoring the significant impact of integrated sentiment analysis on market predictions. These studies collectively highlight the importance of sentiment analysis in understanding and forecasting market dynamics across various financial assets.

\section{Conclusion}
The literature review underscores the multifaceted and challenging nature of predicting stock and cryptocurrency prices, emphasizing the importance of robust predictions for effective trading strategies. Research on stock price prediction is extensive, utilizing various machine learning (ML) and deep learning (DL) methods. Studies such as those by Kara et al. (2011) and Patel (2015) highlight the effectiveness of SVM and random forest models, respectively, in forecasting stock prices using technical indicators. Similarly, Chen (2020) and Weng (2018) demonstrate the predictive power of logistic regression, LSTM networks, and ensemble methods for Bitcoin and stock prices, leveraging comprehensive feature sets that include market variables and sentiment scores. The review also notes the evolving efficiency of cryptocurrency markets, with studies like those by Kang et al. (2022) and Tran (2020) revealing limited adherence to the Efficient Market Hypothesis (EMH), indicating significant information processing inefficiencies.
The impact of sentiment analysis on market predictions emerges as a critical theme. Research by Mai (2018) and Kraaijeveld (2020) establishes the predictive power of social media sentiment on cryptocurrency prices, while Kristoufek (2013) and Chan (2003) demonstrate similar effects for online search activity and news sentiment on Bitcoin and stock markets. Studies such as Rognone (2020) and Khan (2020) further validate the significant influence of real-time sentiment, integrating social media and news data to achieve high prediction accuracy. These findings collectively highlight the importance of incorporating diverse feature sets, including sentiment analysis, to enhance the predictability of financial markets and challenge traditional notions of market efficiency.
